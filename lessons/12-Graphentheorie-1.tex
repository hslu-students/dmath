\section{Graphentheorie 1}

\subsection{(Ecken)grade}

Eckengrad: $ sum_{v \in V} deg(v) = 2 \cdot |E| $
\\
Maximalgrad: $ \Delta(G) = max_{v \in V(G)} {deg}(v) $
\\
Maximalgrad: $ \delta(G) = min_{v \in V(G)} {deg}(v) $

\subsection{Wichtige Graphen}

Vollständiger Graph $K_n$ mit $n$ Knoten: genau eine Kante zwischen je zwei Knoten ($m$ Kanten).
\\
$ m = {{n}\choose{2}} = {{(n-1)n}\over{2}} $ 

\subsection{Baum}

Baum mit $n$ Knoten: $n-1$ Kanten.
\\
Baum mit $i$ inneren Knoten: $n=m \cdot i + 1$ Knoten
\\
$m$-facher Baum der Höhe $h$: höchstens $m^h$ Blätter.

\subsection{Page-Rank-Algorithmus}

Gewicht der Seite $PR_i$ in einem Netz mit $N$ Seiten, Dämpfungsfaktor $d$ ($[0;1]$), $C_j$ von Seite $j$ abgehende Links:
\\
$PR_i = {{1-d}\over{n}} + d \cdot \sum_{j} {{PR_j}\over{C_j}}$

\subsection{Matrizen}

$n$ Ecken, $m$ Kanten

\begin{itemize}
\item{Adjazenzmatrix $A(G)$: $n \times n$-Matrix (Knoten/Knoten) mit Anzahl Kanten zwischen den Ecken.}
\item{Inzidenzmatrix $B(G)$: $n \times m$-Matrix (Knoten/Kanten) mit $1$ (Knoten liegt auf Kante) oder $0$ (Knoten \textit{nicht} auf Kante)}
\item{Gradmatrix $D(G)$: $n \times n$-Diagonal-Matrix (Knoten/Knoten), Grade der Knoten auf der Diagonalen}
\end{itemize}

\subsection{Wege und Kreise}

Anzahl Wege der Länge $l$ von Knoten $i$ zu $j$: Eintrag $(i,j)$ von $A(G)^l$ (Adjazenzmatrix hoch $l$)

\begin{itemize}
\item{Weg: Folge von Kanten $e_1={a,b},e_2={b,c},\dots$}
\item{Kreis: Weg mit übereinstimmendem Anfangs- und Endpunkt (Länge $>0$)}
\item{einfacher Kreis: jede Kante kommt höchstens einmal vor}
\item{Eulerweg: Weg, der jede Kante einmal durchläuft}
\item{Eulerkreis: Kreis, der jede Kante einmal durchläuft}
\item{Hamiltonweg: Weg, der jeden Knoten einmal durchläuft}
\item{Hamiltonkreis: Kreis, der jeden Knoten einmal durchläuft}
\item{Satz von Dirac: ein Graph mit $n \geq 3$ Knoten mit Grad $\geq n/2$ hat einen Hamiltonkreis.}
\item{Satz von Ore: ein Graph mit $n \geq 3$ mit ${deg}(v)+{deg(u) \geq n}$ für jedes Paar $u,v$ von nicht benachbarten Ecken hat einen Hamiltonkreis.}
\end{itemize}
