\section{Foundations}

\subsection{Operationen}
\begin{tabular}{lll}
	Negation & $\neg p$ & \textit{Verneinung} \\ 
	Konkunktion & $p \wedge q$ & \textit{Und-Verknüpfung} \\ 
	Disjunktion & $p \vee q$ & \textit{Oder-Verknüpfung} \\
	EXOR & $p \oplus q$ & \textit{Exklusiv-Oder} \\
	Implikation & $p \rightarrow q$ & \textit{falls p dann q} \\
	Bikonditional & $p \leftrightarrow q$ & \textit{p genau dann wenn q} \\
\end{tabular} 

\subsection{Prioritäten der Operationen}
\begin{tabular}{cccccc}
	$\neg$ & $\wedge$ & $\vee$ & $\oplus$ & $\rightarrow$ & $\leftrightarrow$ \\
	1 & 2 & 3 & 4 & 5 & 6
\end{tabular} 

\subsection{Tautologie \& Kontraktion}
\begin{tabular}{lll}
	Tautologie & $p \vee \neg p$ & \textit{immer wahre Aussage} \\ 
	Kontraktion & $p \wedge \neg q$ & \textit{immer falsche Aussage} \\ 
\end{tabular} 

\subsection{Logische Äquivalenzgesetze}
\begin{tabular}{lll}
	Identität & $p \wedge \textbf{T} \equiv p$ & $p \vee \textbf{F} \equiv p$ \\
	Dominanz & $p \vee \textbf{T} \equiv \textbf{T}$ & $p \wedge \textbf{F} \equiv \textbf{F}$ \\
	Negation & $p \vee \neg p \equiv \textbf{T}$ & $p \wedge \neg p \equiv \textbf{F}$ \\
	Assoziativ 1 & \multicolumn{2}{l}{$(p \vee q) \vee r \equiv p \vee (q \vee r)$} \\ 
	Assoziativ 2 & \multicolumn{2}{l}{$(p \wedge q) \wedge r \equiv p \wedge (q \wedge r)$} \\ 
	Distributiv 1 & \multicolumn{2}{l}{$p \vee (q \wedge r) \equiv (p \vee q) \wedge (p \vee r)$} \\ 
	Distributiv 2 & \multicolumn{2}{l}{$p \wedge (q \vee r) \equiv (p \wedge q) \vee (p \wedge r)$} \\ 
	De Morgan’s 1 &
	\multicolumn{2}{l}{$\neg (p \wedge q) \equiv \neg p \vee \neg q$} \\ 
	De Morgan’s 2 & 
	\multicolumn{2}{l}{$\neg (p \vee q) \equiv \neg p \wedge \neg q$} \\ 
\end{tabular} 

\subsection{Äquivalenzgesetze}
\begin{tabular}{rcl}
	$p \rightarrow q$ & $\equiv$ & $\neg p \vee q$ \\
	$p \rightarrow q$ & $\equiv$ & $\neg q \rightarrow \neg p$ \\
	$p \vee q$ & $\equiv$ & $\neg p \rightarrow q$ \\
	$p \wedge q$ & $\equiv$ & $\neg (p \rightarrow \neg q)$ \\
	$\neg (p \rightarrow q)$ & $\equiv$ & $p \wedge \neg q$ \\
	\\
	$p \leftrightarrow q$ & $\equiv$ & $(p \rightarrow q) \wedge (q \rightarrow p)$ \\
	$p \leftrightarrow q$ & $\equiv$ & $\neg p \leftrightarrow \neg q$ \\
	$p \leftrightarrow q$ & $\equiv$ & $(p \wedge q) \vee (\neg p \wedge \neg q)$ \\
	$\neg (p \leftrightarrow q)$ & $\equiv$ & $p \leftrightarrow \neg q$ \\
	\\
	$p \rightarrow (q \wedge r)$ & $\equiv$ & $(p \rightarrow q) \wedge (p \rightarrow r)$ \\
	$(p \vee q) \rightarrow r$ & $\equiv$ & $(p \rightarrow r) \wedge (q \rightarrow r)$ \\
	$p \rightarrow (q \vee r)$ & $\equiv$ & $(p \rightarrow q) \vee (p \rightarrow r)$ \\
	$(p \wedge q) \rightarrow r$ & $\equiv$ & $(p \rightarrow r) \vee (q \rightarrow r)$ \\
	\\
	$p \oplus q$ & $\equiv$ & $(p \vee q) \wedge (\neg p \vee \neg q)$ \\
	$\neg (p \oplus q)$ & $\equiv$ & $(p \wedge q) \vee (\neg p \wedge \neg q)$ \\
	$\neg (p \oplus q)$ & $\equiv$ & $p \leftrightarrow q$ \\
\end{tabular}

\subsection{Quantifikatoren}
\begin{tabular}{lrl}
	For All & $\forall$ & \textit{für alle $\textbf{x}$ aus $\textbf{P}$ wahr} \\ 
	Exists & $\exists$ & \textit{für mindestens ein $\textbf{x}$ aus $\textbf{P}$ wahr} \\ 
	Not Exists & $\neg \exists$ & \textit{für alle $\textbf{x}$ aus $\textbf{P}$ falsch} \\ 
	Not For All & $\neg \forall$ & \textit{für mindestens ein $\textbf{x}$ aus $\textbf{P}$ falsch} \\ 
\end{tabular} 

\subsection{Negation von Quantifikatoren}
\begin{tabular}{rcl}
	$\neg \exists x P(x)$ & $\equiv$ & $\forall x \neg P(x)$ \\
	$\neg \forall x P(x)$ & $\equiv$ & $\exists x \neg P(x)$ \\
\end{tabular}

\subsection{Beweise}
\begin{tabular}{ll}
	direkter Beweis & $p \rightarrow q$ \\ 
	indirekter Beweis & $\neg q \rightarrow \neg p$ \\ 
	Widerspruch & $\neg p \rightarrow q$ \\
	\textit{Vorgehen Widerspruch}& $(\neg p \rightarrow \textbf{f}) \Rightarrow (p \rightarrow \textbf{w})$ \\
\end{tabular} 

